\documentclass{article}

\usepackage{indentfirst}
\usepackage{amsmath}
\usepackage{geometry}
\usepackage{amssymb}

\geometry{a4paper,scale=0.9}


\title{HW1}
\author{Sicheng Wang}


\begin{document}

	\maketitle
	\section{PROBLEM 3}

	\subsection{DHS chapter8, Pb1.}
		Suppose a decision tree has a path with repeated splits $s_1$ and $s_2$, where $s_1 = s_2 = (f, t)$.\par
		Say that $s_1$ is prior to $s_2$ ($s_1$ is closer to the root).\par
		Suppose at split $s_1$, this path goes to the left branch, which means all the data subject to this path have feature $f$ less than threshold $t$.\par
		Then, at split $s_2$, these data will again all fall into the left path. So $s_2$ has an empty right branch. Therefore, removing $s_2$ from this path will have no impact on the decision result.\par
		Similarly, if this path goes to the right branch at $s_1$, removing $s_2$ from this path will have no impact on the decision result.\par
		Consequently, an equivalent decision tree only with distinct splits on each path can be constructed by iteratively removing all the repeated splits which is closer to the leaves.\par

	\subsection{DHS chapter8, 2.(a)}
		For a $B$-way split, if this feature is numeric, then this split can be treated as cutting the feature value range into B sections. In this case, it can be represented with $B-1$ binary splits, from low to high.
		If the feature is categorical, this split can be represented by $B-1$ 'equals' split.
		Therefore a $B$-way split, the same function can be achieved by replacing it with $B-1$ binary splits, which will also produce $B$ possible leaves, each of which corresponding to one of the childe nodes of the $B$-way split.\par
		That is to say,given an arbitraty decision tree, all of its binary non-binary split nodes can be replaced with binary splits. By doing so, this new decision tree is a binary tree because it contains only binary splits.\par

	\subsection{DHS chapter8, 2.(b)}
		For a $B$-way split, there are $B$ possible outcomes, since one binary split will only introduce one more outcome, there needs to be at least $B-1$ nodes. which will take at least $log_2(B-1)$ levels (by splitting a numerical feature space in a binary search fashion).
		As for the upper bound, the can be as many as $B-1$ splits in a row, in a root-leaf path (no repeating split), which can take as much as $B-1$ levels.

	\subsection{DHS chapter8, 2.(c)}
		As demonstrated in b), the lower bound for number of nodes is $B-1$.\par
		The upper bound case is $B-1$, when all thresholds are encountered from low to high in a path (all the other branches are leaf nodes).

 

	\section{PROBLEM 4}
	\subsection{(a)}
		\begin{equation*}
		\begin{aligned}
			\Delta i(N)	= &i(N) - P_L i(N_L) - (1 - P_L)i(N_R)\\
						= &-\Sigma_j [\frac{N_j}{N}log_2(\frac{N_j}{N})] + \frac{N_L}{N} \Sigma_j [ \frac{N_{jL}}{N_L}log_2(\frac{N_{jL}}{N_L})] + \frac{N-N_L}{N} \Sigma_j [ \frac{N_j - N_{jL}}{N-N_L}log_2(\frac{N_j - N_{jL}}{N-N_L}) ]\\
						= &-\frac{1}{N} \Sigma_j \left( [N_j(log_2 N_j - log_2 N) - N_{jL} (log_2 N_{jL} - log_2{N_L}) - (N_j-N_{jL})(log_2 (N_j-N_{jL}) - log_2 (N-N_L))] \right) \\
						= &-\frac{1}{N} \Sigma_j \left( [N_j log_2 N_j - N_{jL} log_2 N_{jL} - (N_j - N_{jL}) log_2 (N_j - N_{jL})]\right) \\
						  &+\frac{1}{N} \Sigma_j \left( [N_j log_2 N - N_{jL} log_2 N_L - (N_j - N_{jL}) log_2 (N - N_L)]\right) \\
		\end{aligned}
		\end{equation*}\par
		The first part is a convex function, which is maximized to $0$ when $N_{jL} = 0$ or $N_{jL} = N_j$, depending on $N_L$ because $\Sigma_i N_{jL} = N_L$.\par
		The second part is a convex function, which is mimimized to $1$ when $N_L = \frac{N}{2}$.\par
		Finally, this equation is maximized to $1$ when the given data is seperated to two subsets with equal size, and there are no labels appear in both subsets.\par
		In another way, "yes/no" query means a binary split, the data points decoded by this point either go to the left branch or the right branch. This infomation takes up to 1 bit to encode(maximum of 1 if they are splited equally and one label never goes to two braches, otherwise it's less informative).

	\subsection{(b)}
		The decrease in entropy impurity provided by a single $B$-way split can never be greater than $log_2B$ bit.


	\section{PROBLEM 5}
		To find the normal equations solution, minimize $$e(a,b) = \Sigma_i\left[(y_i - ax_i - b)^2\right]$$\par
		Take the partial derivative of this equation:
			$$\frac{\partial e(a,b)}{\partial a} = \Sigma_i\left[(y_i - ax_i - b)x_i\right]$$
			$$\frac{\partial e(a,b)}{\partial b} = \Sigma_i\left[(y_i - ax_i - b)\right]$$\par
		To find the minimal, let $\frac{\partial e(a,b)}{\partial a} = \frac{\partial e(a,b)}{\partial b} = 0$
		Then, 
		\begin{equation*}
		\left\{
			\begin{aligned}			
			\Sigma_i (x_i^2) a &+ \Sigma_i x_i b &- \Sigma_i x_iy_i &= 0\\
			\Sigma_i x_i a     &+ n b            &- \Sigma_i y_i    &= 0
			\end{aligned}
		\right.
		\end{equation*}\par
		Solve this equation,
		\begin{equation*}
		\left\{
			\begin{aligned}			
			a &= \frac{n\Sigma_i(x_iy_i) - (\Sigma_i x_i) (\Sigma_i y_i)}{n\Sigma_i (x_i^2) - (\Sigma_i x_i)^2}\\
			b &= \frac{\Sigma_i(y_i)(\Sigma_ix_i)^2 - \Sigma_i(x_iy_i)\Sigma_i(x_i)}{n\Sigma_i (x_i^2) - (\Sigma_i x_i)^2}\\
			\end{aligned}
		\right.
		\end{equation*}\par


	\section{PROBLEM 7}
		Suppose two convex hulls are both linearly separable and intersected.\par
		Because they intersect, there must exits a point that are inside both convex hulls.\par
		
		Let  $x = [x_1\ x_2\ ...\ x_n]$ be a point in the intersecting area, $\alpha^{(1)} = [\alpha_1\ \alpha_2\ ...\ \alpha_n]$ for this point represented with the first convex hull's vectors, and $\alpha^{(2)} = [\alpha_1\ \alpha_2\ ...\ \alpha_n]$ for this point represented with the second convex hull's vectors.
		$$\alpha^{(1)}x = \alpha^{(2)}x$$\par
		Because these two convex hulls are linearly seperable, then we should have:\par
		$\alpha^{(1)}x > 0$ if x is within the first convex hull\par
		and\par
		$\alpha^{(1)}x < 0$ if x is within the second convex hull\par
		This conflicts with the first equation. So two convex hulls cannot be both linearly separable and intersected.



	\section{PROBLEM 8}
		\begin{equation*}
			\begin{aligned}
				 & \triangledown_A tr(ABA^TC)\\
				=& \triangledown_A tr(ABA^TC) (A^T\ treated\ as\ constant) + (\triangledown_{A^T} tr(A^TCAB))^T (A\ treated\ as\ constant)\\
				=& (BA^TC)^T + ((CAB)^T)^T\\
				=& C^TAB^T + CAB
			\end{aligned}
		\end{equation*}

\end{document}
